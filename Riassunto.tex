\section*{Riassunto del contenuto della Tesi}
%\subsection*{Campo applicativo}
La tesi tratta l'applicazione delle reti neurali % al telerilevamento 
% , nello specifico di reti 
% convoluzionali (CNN), 
% Il telerilevamento, è una scienza che permette di identificare, misurare ed
% analizzare le caratteristiche qualitative e quantitative di una 
% determinata area di interesse senza entrare in contatto diretto con 
% essa. Tipicamente, l’oggetto di studio nel telerilevamento è il 
% pianeta Terra in tutte le sue componenti: territorio, acqua e atmosfera.
% Nello specifico, l'applicazione 
% delle reti neurali convoluzionali (CNN) all'agricoltura di precisione, 
per svolgere  
una mappatura delle colture agricole (\textit{Crop mapping}) 
utilizzando immagini satellitari Sentinel-2.
La mappatura delle colture è una delle molte applicazioni del telerilevamento e 
consiste nell’identificazione e classificazione delle colture agricole 
all'interno di un'area geografica.
Questa applicazione è fondamentale per supportare il 
processo decisionale e fornire inventari accurati e tempestivi per stimare la 
produzione e monitorare la crescita dinamica delle colture a varie scale. 
%\subsection*{Parte sperimentale}
La tesi è strutturata su otto capitoli:
Il primo capitolo tratta la definizione di rete neurale, di \textit{Deep learning} e di 
\text{Machine Learning}.
Viene inoltre illustrato il concetto di dataset ed i metodi che si utilizzano per 
valutare le prestazioni di un modello di \textit{Deep learning}.
Nel secondo capitolo vengono presentati i framework che permettono lo sviluppo di modelli 
di \textit{Deep learning}, il linguaggio di programmazione utilizzato per la realizzazione dei 
modelli ed il concetto di Tensore.
Nel terzo capitolo vengono richiamati i concetti fondamentali del
telerilevamento, illustrando come avviene l'acquisizione delle informazioni nel telerilevamento.
Nel quarto capitolo si approfondisce il funzionamento di una rete neurale classica, 
partendo dalle basi fino a trattare concetti più complessi.
Nel quinto capitolo vengono trattate le reti neurali convolutive, illustrando i principi 
su cui si basano queste tipologie di reti e gli elementi di cui sono composte. 
% Nel sesto capitolo vengono richiamati i concetti di base dietro alla segmentazione delle immagini.
% Sempre nello stesso capitolo, viene anche illustrata l'architettura UNet.
Nel sesto capitolo vengono trattati i concetti di base relativi alla segmentazione 
delle immagini. Illustrando anche l'architettura UNet, spesso utilizzata per applicazioni 
di segmentazione semantica delle immagini.
Il settimo capitolo tratta la sperimentazione, illustrando il processo per la realizzazione di 
un modello in grado di eseguire la segmentazione semantica su delle immagini satellitari.
% Mentre nell'ottavo capitolo vengono discussi i risultati ottenuti, mettendoli a confronto con i 
% risultati ottenuti da altri modelli applicati allo stesso problema. 
Nell'ottavo capitolo vengono discussi i risultati ottenuti, confrontandoli
con quelli ottenuti da altri modelli descritti in articoli che utilizzano lo stesso dataset.
Nella parte sperimentale della tesi è stato esposto l'intero 
procedimento per realizzare un sistema capace di apprendere 
automaticamente le caratteristiche di ogni campo agricolo, descrivendo i vari problemi riscontrati 
e le soluzioni adottate.
Per l'addestramento di questo sistema è stato utilizzo il dataset Sentinel2-Munich480, 
un dataset contenente immagini satellitari Sentinel-2 sull'area a nord di Monaco, in Germania.
Per lo sviluppo del modello sono state applicate due diverse architetture di reti neurali per la 
segmentazione semantica di immagini.
La prima versione del sistema era 
basata sull'architettura U-Net, ma presentava alcune limitazioni nel riconoscimento di molte colture agricole.
La seconda versione del sistema si basava invece su una versione dell'architettura U-Net che 
sfruttava le convoluzioni 3D. 
L'utilizzo di quest'ultima architettura, ha permesso al sistema di raggiungere un precisione 
del 91.92\% sul dataset di valutazione ed una precisione del 91.84\% sul dataset 
di validazione, raggiungendo risultati molto vicini allo stato dell'arte nel campo del 
crop mapping sul dataset Sentinel2-Munich480.






% \subsection*{Struttura della tesi}
% Il libro si articola in un'esauriente trattazione tecnica che 
% spazia dai fondamenti teorici dell'intelligenza artificiale (AI) e 
% dell'apprendimento automatico (Machine Learning, ML) fino a 
% specifiche applicazioni nel telerilevamento per la segmentazione 
% delle immagini, con particolare attenzione ai dati 
% satellitari Sentinel-2.

% Introduzione all'Intelligenza Artificiale, Machine Learning e Deep Learning
% Il primo capitolo introduce i concetti fondamentali di intelligenza artificiale, delineando l'evoluzione storica e i principi alla base di questa disciplina. Successivamente, il focus si sposta sul machine learning, descrivendone le definizioni, le modalità di apprendimento (supervisionato, non supervisionato e per rinforzo) e le principali categorie di problemi affrontati. La sezione dedicata al deep learning esplora le caratteristiche delle reti neurali profonde, enfatizzando l'importanza dei dataset e delle tecniche di valutazione dei modelli, come la matrice di confusione, l'accuratezza, precisione, richiamo e F1-score.

% Frameworks e Linguaggi di Programmazione
% Il secondo capitolo analizza i principali strumenti di sviluppo, con un approfondimento sul linguaggio Python e sui framework per il deep learning, come TensorFlow, Keras, PyTorch e PyTorch Lightning. Viene sottolineata l'importanza dei tensori come struttura dati centrale per il calcolo nei modelli neurali, discutendo le loro proprietà computazionali e rappresentazioni.

% Caratterizzazione Remota del Suolo
% La sezione sul telerilevamento (remote sensing) fornisce una panoramica completa delle tecniche e tecnologie impiegate per l'osservazione della Terra. Si analizzano le piattaforme di acquisizione dati, lo spettro elettromagnetico, le interazioni tra radiazione e superficie terrestre e le firme spettrali. Ampio spazio è dedicato al programma Copernicus e alle missioni Sentinel-2, spiegandone le caratteristiche tecniche e le applicazioni.

% Reti Neurali Artificiali
% Questo capitolo affronta la struttura e il funzionamento delle reti neurali artificiali (ANN). Dopo un'introduzione alle reti biologiche, si passa ai modelli matematici fondamentali, come il modello McCulloch-Pitts, il perceptrone e le reti multilivello (MLP). Sono trattati dettagliatamente concetti come il gradient descent, la backpropagation e le funzioni di attivazione, inclusi Softmax e ReLU. La sezione si conclude con una discussione sugli ottimizzatori e sulla cross-entropy come funzione di perdita.

% Reti Neurali Convoluzionali (CNN)
% Le CNN vengono introdotte con una descrizione dei principi matematici della convoluzione discreta e della cross-correlation. Si approfondiscono le peculiarità delle reti convoluzionali, i parametri della convoluzione e le tecniche di pooling (Max e Average Pooling).

% Image Segmentation e Architettura U-Net
% Il capitolo sulla segmentazione delle immagini presenta le diverse tecniche per suddividere un'immagine in regioni significative, con particolare attenzione all'architettura U-Net. Vengono spiegati gli aspetti strutturali delle U-Net, come le skip connections e l'up-convolution, che rendono questo modello particolarmente adatto per applicazioni di segmentazione.

% Sperimentazione e Applicazioni
% La parte finale si concentra su esperimenti pratici. Viene descritto l'approccio al riconoscimento di campi agricoli utilizzando i dati Sentinel-2. Le sezioni includono dettagli sull'implementazione delle reti neurali (incluso l'uso di U-Net e di convoluzioni 3D), sull'elaborazione di dataset multispettrali e sull'ottimizzazione dei risultati tramite tecniche come la mosaicatura e l'esclusione di classi irrilevanti.

% Conclusioni
% Il libro offre un'analisi approfondita e multidisciplinare, integrando teorie di base, strumenti pratici e applicazioni avanzate. È una risorsa ideale per chi desidera comprendere il ruolo delle reti neurali nella segmentazione di immagini, con un particolare focus sul telerilevamento e sull'elaborazione di dati multispettrali.