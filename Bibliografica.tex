
\renewcommand{\bibname}{Bibliografia}
% \begin{thebibliography}{99}

%     \bibitem{author2024}
%     Autore Nome, *Titolo del libro o articolo*, Editore, Anno.
    
%     \bibitem{smith2018}
%     Smith J., "An Example Article Title", *Journal of LaTeX Examples*, vol. 42, no. 3, pp. 123-456, 2018.
    
%     \bibitem{website}
%     Titolo del sito web. [Online]. Disponibile: %\url{http://www.esempio.com}
    
% \end{thebibliography}

\begin{thebibliography}{99}

\setlength{\itemsep}{0pt} % Riduce lo spazio verticale tra le voci
\setlength{\parskip}{0pt} % Elimina eventuale spazio extra tra i paragrafi
\raggedright % Allinea tutto a sinistra
\sloppy % Aumenta la tolleranza per la giustificazione


% ========================= [MATEMATICA] =========================
\bibitem{prodotto_Hadamard}
\url{https://en.wikipedia.org/wiki/Hadamard_product_(matrices)}

\bibitem{Derivate_parziali}
\url{https://it.wikipedia.org/wiki/Derivata_parziale}

%==================================================================

% --- IA, ML e DL ---
\bibitem{IA_ML_DL}
\url{https://pmf-research.eu/deep-learning-machine-learning-ia-tradizionale/}.

\bibitem{IA_1}
\url{https://it.wikipedia.org/wiki/Intelligenza_artificiale}

\bibitem{INGLOBAZIONE_IA}
\url{https://www.digitalbimitalia.it/it/news/artificial-intelligence-machine-learning-o-deep-learning-possibili-applicazioni-allecosistema-bim/}

\bibitem{IA_EM_SH}
\url{https://www.bbc.com/news/technology-30290540}

\bibitem{ASPETTI_ML}
\url{https://www.ibm.com/it-it/topics/machine-learning}

\bibitem{ASPETTI_DEEP_LEARNING}
\url{https://www.agendadigitale.eu/cultura-digitale/deep-learning-cose-come-funziona-e-applicazioni}

\bibitem{ASPETTI_DEEP_LEARNING_2}
\url{https://www.bnova.it/intelligenza-artificiale/deep-learning-cose-e-quali-le-applicazioni/}

\bibitem{I_3_PROBLEMI_ML_e_APPRENDIMENTO}
\url{https://medium.com/%40jacopokahl/i-tre-principali-tipi-di-machine-learning-77ca20d0dbdd}

% *************************************************************
% ========================= [DATASET] =========================
% *************************************************************
% definizione
\bibitem{Dataset_Bright}
\url{https://medium.com/@Bright-Data/what-is-a-dataset-definition-use-cases-benefits-and-example-9aaf5ecc301e}.

\bibitem{Dataset_Wikipedia}
\url{https://it.wikipedia.org/wiki/Dataset}.

\bibitem{UTILIZZI_DATASET}
\url{https://en.wikipedia.org/wiki/Training%2C_validation%2C_and_test_data_sets}

\bibitem{DIVISIONE_DATASET}
\url{https://encord.com/blog/train-val-test-split}

\bibitem{DIVISIONI_DATASET}
\url{https://developers.google.com/machine-learning/crash-course/overfitting/dividing-datasets?hl=it}

\bibitem{DIVISIONE_DATASET_2}
\url{https://deeplearningitalia.com/parliamo-di-split-utilizzati-per-dividere-il-dataset-in-training-validation-e-test-nel-machine-learning/}

% --- MNIST ---
\bibitem{MNIST_Kaggle}
\url{https://www.kaggle.com/datasets/hojjatk/mnist-dataset}.

\bibitem{MNIST_Analysis}
\url{https://en.wikipedia.org/wiki/MNIST_database}

% --- Munich480 ---
\bibitem{Munich480}
\url{https://www.kaggle.com/datasets/artelabsuper/sentinel2-munich480}

\bibitem{Monaco}
\url{https://it.wikipedia.org/wiki/Monaco_di_Baviera}


% *************************************************************
% ========================= [VALUTAZIONE] =========================
% *************************************************************
% --- Matrice di confusione ---
\bibitem{Confusion_Matrix_e_metrics1}
\url{https://www.ibm.com/it-it/topics/confusion-matrix}

\bibitem{Confusion_Matrix_e_metrics2}
\url{https://www.zerounoweb.it/big-data/confusion-matrix-guida-pratica-per-valutare-il-modello-di-classificazione/}.

% --- Generale ---
\bibitem{Python_Wikipedia}
\url{https://it.wikipedia.org/wiki/Python}.

% --- Framework ---
\bibitem{Framework_AnalyticsVidhya}
\url{https://www.analyticsvidhya.com/blog/2019/03/deep-learning-frameworks-comparison/}.
\bibitem{Framework_VisoAI}
\url{https://viso.ai/deep-learning/pytorch-vs-tensorflow/}.
\bibitem{Framework_Devopedia}
\url{https://devopedia.org/deep-learning-frameworks}.
\bibitem{Framework_PapersWithCode}
\url{https://paperswithcode.com/trends}.

% --- PyTorch Lightning ---
\bibitem{PyTorchLightning}
\url{https://en.wikipedia.org/wiki/PyTorch_Lightning}.

\bibitem{PyTorchLightning_site}
\url{https://lightning.ai/docs/pytorch/stable/}

% --- Tensori ---
\bibitem{Tensore_Fidacaro}
\url{https://fidacaro.com/cose-un-tensore-e-come-viene-utilizzato-nel-machine-learning}.
\bibitem{Tensore_IBM}
\url{https://www.ibm.com/it-it/topics/pytorch}.

\bibitem{tensor_analytic}
\url{https://www.analyticsvidhya.com/blog/2022/07/data-representation-in-neural-networks-tensor/}

\bibitem{tensor_medium1}
\url{https://towardsdatascience.com/what-is-a-tensor-in-deep-learning-6dedd95d6507}

\bibitem{tensor_medium2}
https://medium.datadriveninvestor.com/what-is-the-tensor-in-deep-learning-77c2af7224a1

\bibitem{tensor_strano}
\url{https://deepai.org/machine-learning-glossary-and-terms}


% *************************************************************
% ========================= [ELEMENTI_DEEP_LEARNING] =========================
% *************************************************************

\bibitem{PARAGONE_CERVELLE_RETE_NEURALE_1}
\url{https://www.ibm.com/it-it/topics/neural-networks}

\bibitem{PARAGONE_CERVELLE_RETE_NEURALE_2}
\url{https://www.missionescienza.it/reti-neurali-artificiali-parte-2/}

\bibitem{NEURONE_BIOLOGICO}
\url{https://it.m.wikipedia.org/wiki/Rete_neurale}

\bibitem{IMAMGINE_MODELLO_MAT_NEURONE_GENERALE_PERCETTONE}
\url{https://the-geeks-of-the-round-table.medium.com/introduction-to-deep-learning-the-perceptron-part-2-bccf30be1a4b}

\bibitem{MODELLO_E_FUNZIONAMNETO_REURONE_BIOLOGICO}
\url{https://medium.com/@siddharthshah2601/mcculloch-pitts-neuron-a-computational-model-of-biological-neuron-ce57239a951e}


\bibitem{STORIA_PERCETTONE}
https://medium.com/@nexomind/alle-origini-del-neurone-artificiale-31f848efd6b6

\bibitem{LIMITI_PERCETTRONE}
\url{https://medium.com/@vnohitha13/neural-networks-52bec25688eb}

\bibitem{ALL_SLP}   
https://vitalflux.com/how-do-we-build-deep-neural-network-using-perceptron/

\bibitem{IMMAGINE_PERCETTRONE_ASPETTI}
\url{https://medium.com/@rdugue1/neural-network-building-blocks-7ea6f8c790bf}

\bibitem{FORMULE_PERCETTRONE}
\url{https://medium.com/codex/single-layer-perceptron-and-activation-function-b6b74b4aae66}

\bibitem{APPRENDIMENTO_PRECETTRONE}
\url{https://www.saedsayad.com/artificial_neural_network_bkp.htm}

\bibitem{ASPETTI_APPRENDIMENTO_PERCETTONE}
https://www.geeksforgeeks.org/what-is-perceptron-the-simplest-artificial-neural-network/

\bibitem{ASPETTI_MLP_1}
\url{https://wiki.pathmind.com/multilayer-perceptron}

\bibitem{ASPETTI_MLP_2}
https://www.datacamp.com/tutorial/multilayer-perceptrons-in-machine-learning

% --- Gradient Descent ---
\bibitem{GradientDescent_NeuralNetworks}
\url{http://neuralnetworksanddeeplearning.com/}.

\bibitem{GradientDescent_TowardsDataScience}
\url{https://towardsdatascience.com/gradient-descent-algorithm-a-deep-dive-cf04e8115f21}.

\bibitem{GradientDescent_Medium}
https://medium.com/geekculture/mathematics-behind-gradient-descent-f2a49a0b714f

\bibitem{Immagine_MLP}
\url{https://www.mondadorieducation.it/fisica-scientifica-ss2/le-reti-neurali-artificiali/}

%backpropagation

\bibitem{Introduzione_backpropagation}
\url{https://towardsdatascience.com/introduction-to-math-behind-neural-networks-e8b60dbbdeba}

% --- Cross-Entropy ---
\bibitem{CrossEntropy_Wikipedia}
\url{https://it.wikipedia.org/wiki/Entropia_(teoria_dell%27informazione)}.

\bibitem{CrossEntropy_365DataScience}
\url{https://365datascience.com/tutorials/machine-learning-tutorials/cross-entropy-loss/?utm_source=chatgpt.com}.

\bibitem{CrossEntropy_DataCamp}
\url{https://www.datacamp.com/tutorial/the-cross-entropy-loss-function-in-machine-learning}.

% --- Funzioni di Attivazione ---
\bibitem{ActivationFunctions_NetAI}
\url{https://netai.it/guida-rapida-alle-funzioni-di-attivazione-nel-deep-learning/#comment-27}.

\bibitem{ActivationFunctions_MEDIUM}
\url{https://medium.com/analytics-vidhya/activation-functions-all-you-need-to-know-355a850d025e}

\bibitem{ActivationFunctions_Softmax}
\url{https://neuralthreads.medium.com/softmax-function-it-is-frustrating-that-everyone-talks-about-it-but-very-few-talk-about-its-54c90b9d0acd}

%Learning rate e optimizer
\bibitem{LearningRate_optimizer}
\url{https://musstafa0804.medium.com/optimizers-in-deep-learning-7bf81fed78a0}

%Full Batch vs Stochastic vs Mini Batch
\bibitem{Full Batch vs Stochastic vs Mini Batch}
\url{https://sweta-nit.medium.com/batch-mini-batch-and-stochastic-gradient-descent-e9bc4cacd461}

%***************************************
%CONVOLUZIONE

%OPERATORE
\bibitem{Definizione_convoluzione_int}
\url{https://en.wikipedia.org/wiki/Convolution}


\bibitem{CONV_PYTORCH}
https://www.geeksforgeeks.org/apply-a-2d-convolution-operation-in-pytorch/

\bibitem{ELEMENTI_CNN_1}
\url{https://medium.com/codex/understanding-convolutional-neural-networks-a-beginners-journey-into-the-architecture-aab30dface10}

\bibitem{ELEMENTI_CNN_2}
\url{https://ujjwalkarn.me/2016/08/11/intuitive-explanation-convnets/}

\bibitem{ELEMENTI_CNN_3}
\url{URhttps://dennybritz.com/posts/wildml/understanding-convolutional-neural-networks-for-nlp/L}

\bibitem{PULLING}
\url{https://www.baeldung.com/cs/neural-networks-pooling-layers}
% \bibitem{DILATATION_3D_CONV}
% \url{https://www.kaggle.com/code/shivamb/3d-convolutions-understanding-use-case}

%LaNet5
\bibitem{LaNet5}
\url{https://www.analyticsvidhya.com/blog/2021/03/the-architecture-of-lenet-5/}

\bibitem{LANet5_Img}
\url{https://en.wikipedia.org/wiki/LeNet}

%aspetti conv
\bibitem{ASPETTI_CONVOLUZIONE_1}
\url{https://www.kaggle.com/code/shivamb/3d-convolutions-understanding-use-case}

\bibitem{ASPETTI_CONVOLUZIONE_2}
\url{https://towardsdatascience.com/intuitively-understanding-convolutions-for-deep-learning-1f6f42faee1}

%immagine kernel movimento
\bibitem{MOVING_3D_Kernel}
\url{https://sarosijbose.github.io/files/talks/A%20General%20Overview%20of%203D%20Convolution%20.pdf}

\bibitem{Paragone_size_3D_kernel}
https://ai.stackexchange.com/questions/13692/when-should-i-use-3d-convolutions

%batch normalization
\bibitem{Batch_Normalization}
\url{https://medium.com/@pouyahallaj/batch-normalization-a-deep-dive-for-machine-learning-enthusiasts-60e4a76d10e8}


% *************************************************************
% ========================= [Remote sensing] =========================
% *************************************************************
\bibitem{ALL1_REMOTE_SENSING}
\url{https://semiautomaticclassificationmanual-v5.readthedocs.io/it/latest/remote_sensing.html}

\bibitem{ALL2_REMOTE_SENSING}
\url{https://www.alspergis.altervista.org/lezione/}

\bibitem{ALL3_REMOTE_SENSING}
\url{https://it.wikipedia.org/wiki/Telerilevamento}

\bibitem{ALL4_REMOTE_SENSING}
\url{https://www.nateko.lu.se/sites/nateko.lu.se.sv/files/remote_sensing_and_gis_20111212.pdf}

\bibitem{ALL5_REMOTE_SENSING}
\url{https://appliedsciences.nasa.gov/sites/default/files/2022-11/Fundamentals_of_RS_Edited_SC.pdf}

\bibitem{ALL6_REMOTE_SENSING}
\url{https://books.google.it/books?id=NkLmDjSS8TsC&printsec=frontcover&hl=it&source=gbs_ge_summary_r&cad=0#v=onepage&q&f=false}

\bibitem{GISGeography_RemoteSensing}
\url{https://gisgeography.com/remote-sensing-earth-observation-guide}

%Rappresentazione delle piattaforme
\bibitem{Rappresentazione_piattaforme}
https://www.researchgate.net/figure/Common-Remote-Sensing-Platform-and-
Sensor-Combinations-and-Remote-Sensing-Data-Left\_fig1\_341582585

\bibitem{Descrizione_Piattaforme}
\url{https://www.spatialpost.com/types-of-platforms-in-remote-sensing/}

%funzionamento della luce
\bibitem{Funzionamento_Luce}
\url{https://it.wikipedia.org/wiki/Luce}

%Onde electro magnetiche
\bibitem{OndeEletroMagnetiche}
\url{https://en.wikipedia.org/wiki/Electromagnetic_radiation}

\bibitem{Onda_IMG}
\url{https://www.rfcafe.com/references/electrical/electromagnetic-wave-physics.htm}

\bibitem{SPETTRO_IMG}
\url{https://geolearn.in/classification-methods-in-remote-sensing-gis/}

%spettro
\bibitem{spetto_magnetico}
\url{https://it.wikipedia.org/wiki/Spettro_elettromagnetico}

%firma spettarle
\bibitem{Firma_spettare}
\url{https://it.wikipedia.org/wiki/Firma_spettrale}

\bibitem{fenomeni_luce}
\url{https://associazionegioconda.it/la-propagazione-della-luce-2}
\bibitem{fenomeni_luce_2}
\url{https://it.wikipedia.org/wiki/Luce}

\bibitem{INTERAZIONI_ONDE}
\url{https://ltb.itc.utwente.nl/498/concept/81854}

\bibitem{INTERAZIONI}
\url{https://ecampusontario.pressbooks.pub/remotesensing/chapter/chapter-4-emr-interactions-with-the-atmosphere-and-with-the-surface/}

\bibitem{Riflettanza}
\url{https://it.wikipedia.org/wiki/Riflettanza}

\bibitem{Radianza}
\url{https://it.wikipedia.org/wiki/Radianza}

\bibitem{RADIANZA_IMG}
\url{https://www.slideshare.net/slideshow/il-corpo-nero-e-la-quantizzazione-dellenergia/36910588}


%copernicus
\bibitem{COPERNICUS_INFO}
\url{https://www.copernicus.eu/it/informazioni-su-copernicus}

%orbita polare
\bibitem{ORBITA_POLARE}
\url{https://it.wikipedia.org/wiki/Orbita_polare}

%eliosincrona
\bibitem{ELIO_SINCRONA}
\url{https://it.wikipedia.org/wiki/Orbita_eliosincrona}





%sentinel 2
\bibitem{SENTINEL2_1_e_bands}
\url{https://gisgeography.com/sentinel-2-bands-combinations/}

%combinazione bande sentinel
\bibitem{COMBINAZIONE_BANDE_SENTINEL2}
\url{https://sentiwiki.copernicus.eu/web/s2-applications}

\bibitem{SENTINEL2_2}
\url{https://it.wikipedia.org/wiki/Sentinel-2}

\bibitem{SENTINEL2_3}
\url{https://en.wikipedia.org/wiki/Sentinel-2}

\bibitem{MSI_SENTINEL2}
\url{https://www.alspergis.altervista.org/data/sentinel2.html}

\bibitem{Tabella_bande}
\url{https://www.satimagingcorp.com/satellite-sensors/other-satellite-sensors/sentinel-2a/}

%sentinel caratteristiche
\bibitem{ALL_ABOUT_SENTINEL2}
\url{https://sentiwiki.copernicus.eu/web/s2-mission}


\bibitem{Immagini_multispettrali}
\url{https://www.edmundoptics.com/knowledge-center/application-notes/imaging/hyperspectral-and-multispectral-imaging/?srsltid=AfmBOorZC8mMEPOWCUTBWsdLJLlQBYNeGG1ZIxcyWoO3PNdRGotbEGht}

\bibitem{immagini_multispettrali2}
\url{https://innoter.com/en/articles/multispectral-imaging/}

\bibitem{immagini_multispettrale3}
\url{https://en.wikipedia.org/wiki/Multispectral_imaging}

\bibitem{immagini_multispettrali_a_colori}
\url{https://www.alspergis.altervista.org/lezione/13.html}

% *************************************************************
% ========================= [Image Segmentation] =========================
% *************************************************************

\bibitem{SPAZIO_LATENTE}
\url{https://www.domsoria.com/2022/11/che-cosa-e-lo-spazio-latente-o-latent-space-ed-a-cosa-serve/}



\bibitem{ImageSegmentation_Provino}
\url{https://www.andreaprovino.it/image-segmentation-segmentazione-semantica-e-delle-istanze}.

\bibitem{ImageSegmentation_Gradient}
\url{https://thegradient.pub/semantic-segmentation/}.

\bibitem{ImageSegmentation_Labeller}
\url{https://www.labellerr.com/blog/semantic-vs-instance-vs-panoptic-which-image-segmentation-technique-to-choose/}.

\bibitem{Imagesegmentation_pulapakura}
\url{https://medium.com/@raj.pulapakura/image-segmentation-a-beginners-guide-0ede91052db7}

%*************UNet***********

\bibitem{APSETTI_UNET}
\url{https://medium.com/@alejandro.itoaramendia/decoding-the-u-net-a-complete-guide-810b1c6d56d8}

\bibitem{Agricoltura_precisione}
\url{https://it.wikipedia.org/wiki/Agricoltura_di_precisione}

\bibitem{Articolo2_3D_UNET}
\url{https://miccai-sb.github.io/materials/Hoang2019b.pdf}

\bibitem{ARTICOLO_ORIGINALE_UNET}
\url{https://arxiv.org/abs/1505.04597}

\bibitem{Implementazione1_3D_UNET}
\url{https://github.com/wolny/pytorch-3dunet/blob/master/pytorch3dunet}

\bibitem{Implementazione2_3D_UNET}
\url{https://gitlab.com/mattiagatti/sentinel2-crop-mapping-models}

\bibitem{RETE_RICORRENTE}
\url{https://it.wikipedia.org/wiki/Rete_neurale_ricorrente}

\bibitem{TRANSFORMER}
\url{https://it.wikipedia.org/wiki/Trasformatore_(informatica)}

\hspace{0.10cm}
\bibitem{ALL_DEEP_LEARNING}
Goodfellow, I., Bengio, Y., \& Courville, A. (2016). Deep Learning. MIT Press. 
\url{https://www.deeplearningbook.org/}.

%articoli UNet 3D
\hspace{0.10cm}
\bibitem{Articolo1_3D_UNET}
Çiçek, Ö., Abdulkadir, A., Lienkamp, S. S., Brox, T., \& Ronneberger, O. (2016). 
"3D U-Net: Learning Dense Volumetric Segmentation from Sparse Annotation". 
In *Medical Image Computing and Computer-Assisted Intervention – MICCAI 2016*. 
Fonte: \url{https://arxiv.org/pdf/1606.06650}

\hspace{0.10cm}
\bibitem{Articolo_backpropagation}
Rumelhart, Hinton e Williams (1986): "Learning 
representations by back-propagating errors"
Disponibile online: \url{https://www.nature.com/articles/323533a0}

\hspace{0.10cm}
\bibitem{ARTICOLO_PERCETTONE}
https://websites.umass.edu/brain-wars/1957-the-birth-of-cognitive-science/the-perceptron-a-perceiving-and-recognizing-automaton/

\hspace{0.10cm}
\bibitem{ARTICOLO_LIMITI_PRECETTRONE}
Minsky, M., \& Papert, S. A. (1969). Perceptrons: An introduction to 
computational geometry. MIT Press.
\url{https://direct.mit.edu/books/monograph/3132/PerceptronsAn-Introduction-to-Computational}

%Multi-Temporal Land Cover Classification with
%Sequential Recurrent Encoders
\newpage
\hspace{0.10cm}
\bibitem{ARTICOLO_ORIGINALE_MUNICH}
Rußwurm, M., \& Körner, M. (2018). 
"Multi-Temporal Land Cover Classification with Sequential Recurrent Encoders". 
In *ISPRS International Journal of Geo-Information*. 
Fonte: \url{https://arxiv.org/pdf/1802.02080}

%ENHANCING CROP SEGMENTATION IN SATELLITE IMAGE TIME
%SERIES WITH TRANSFORMER NETWORKS

\hspace{0.10cm}
\bibitem{ARTICOLO_TRANSFORMER}
Gallo, I., Gatti, M., Landro, N., Loschiavo, C., Rehman, A. U., \& Boschetti, M. (2024). 
"Enhancing Crop Segmentation in Satellite Image Time Series with Transformer Networks". 
In *Remote Sensing Letters*.
Fonte: \url{https://arxiv.org/pdf/2412.01944}

\hspace{0.10cm}
\bibitem{UNET_3D_munich_application}
Ignazio Gallo, Luigi Ranghetti, Nicola Landro, Riccardo La Grassa, and Mirco Boschetti. In-season and dynamic
crop mapping using 3d convolution neural networks and sentinel-2 time series. ISPRS Journal of Photogrammetry
and Remote Sensing, 195:335–352, 2023.

\hspace{0.10cm}
\bibitem{FPN3D}
Ignazio Gallo, Riccardo La Grassa, Nicola Landro, and Mirco Boschetti. Sentinel 2 time series analysis with
3d feature pyramid network and time domain class activation intervals for crop mapping. ISPRS International
Journal of Geo-Information, 10(7), 2021

\end{thebibliography}

