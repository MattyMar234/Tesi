\chapter{Reti Neurali Convoluzionali (CNN)}
Proposta una panoramica molto generale sulle Reti Neurali, si può adesso scendere più
nel dettaglio analizzando quelle che sono le \textbf{reti convoluzionali}.
Il nome “rete neurale convoluzionale” indica una tipologia di rete neurale che impiega 
un’operazione matematica lineare chiamata appunto \textbf{convoluzione}.
Le Reti Neurali Convoluzionali (CNN o Convolutional Neural Networks) sono un tipo di rete 
neurale particolarmente efficace per l'elaborazione di dati strutturati in forma di 
griglie, come le immagini. Sono ampiamente utilizzati in compiti di visione artificiale, 
come il riconoscimento di immagini, la classificazione di oggetti, il rilevamento di volti  
e in molte altre applicazioni.
Le CNN sono progettate per riconoscere dei pattern visivi in modo diretto e non richiedo
no molto preprocessing o comunque ne richiedono una quantità molto limitata; si ispirano 
al modello della corteccia visiva animale: i singoli neuroni in questa parte del cervello 
rispondo solamente a stimoli relativi ad una zona ristretta del campo di osservazione detto
campo recettivo.

%Solitamente, l'operazione utilizzata in una rete neurale convoluzionale non corrisponde esattamente alla definizione di convoluzione utilizzata in altri campi, come l'ingegneria o la matematica pura.
