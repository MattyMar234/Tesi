\chapter{Conclusioni}
% Questa sperimentazione ci ha consentito di capire come implementare e 
% ottimizzare delle reti neurali per la mappatura dei campi agricoli a partire da immagini 
% satellitari sentinel-2. 
% In particolare, sono state sperimentate due diverse architetture, descrivendo per 
% ciascuna il procedimento adottato per migliorarne la precisione e analizzandone i limiti.
% Conducendo l'addestramento e valutazione di ogni modello è esclusivamente sui dati 
% relativi all'anno 2016.
% Tra i modelli sviluppati, il migliore risultato è stato ottenuto dal modello basato 
% sull'architettura UNet-3D. Sul dataset di validazione, 
% questo modello ha raggiunto i seguenti risultati:
Questa sperimentazione ci ha permesso di comprendere come implementare ed 
ottimizzare reti neurali per la mappatura dei campi agricoli utilizzando 
immagini satellitari Sentinel-2.
In particolare, sono state testate due diverse architetture, descrivendo per 
ciascuna il processo adottato per migliorarne la precisione. 
% L'addestramento e la 
% valutazione di ogni modello sono stati effettuati esclusivamente sui dati relativi all'anno 2016.
Tra i modelli sviluppati, i risultati migliori sono stati ottenuti dal modello 
basato sull'architettura UNet-3D. Su un dataset di validazione, questo modello ha 
raggiunto i seguenti risultati:

% \begin{table}[H]
%     \centering
%     \setlength{\tabcolsep}{5pt} % Riduce la spaziatura tra le colonne
%     \renewcommand{\arraystretch}{1.2} % Aumenta lo spazio tra le righe
%     \begin{tabular}{|l||c|c|c|c|c||}
%     \hline
%     \multicolumn{6}{|c|}{\textbf{Validation}} \\
%     \hline
%     \textbf{Class} & \textbf{Precision} & \textbf{Recall} & \textbf{F1-Score} & \textbf{Kappa} & \textbf{\#Pixels} \\
%     \hline
%     Sugar beet    & 86.37 & 93.51 & 88.85 & 0.88 & 38k \\
%     Summer oat    & 69.55 & 75.66 & 70.12 & 0.69 & 26k \\
%     Meadow        & 77.90 & 77.48 & 75.10 & 0.70 & 148k \\
%     Rape          & 89.79 & 93.06 & 90.94 & 0.90 & 82k \\
%     Hop           & 85.46 & 91.22 & 86.62 & 0.81 & 31k \\
%     Winter spelt  & 71.31 & 63.33 & 63.45 & 0.60 & 24k \\
%     Winter triticale & 66.92 & 55.66 & 55.99 & 0.53 & 38k \\
%     Beans         & 76.44 & 84.39 & 78.35 & 0.74 & 18k \\
%     Peas          & 82.81 & 79.52 & 78.31 & 0.77 & 5k  \\
%     Potato        & 84.58 & 85.88 & 83.77 & 0.81 & 78k \\
%     Soybeans      & 81.53 & 83.95 & 80.61 & 0.78 & 12k \\
%     Asparagus     & 79.17 & 78.92 & 75.87 & 0.72 & 13k \\
%     Winter wheat  & 92.12 & 85.07 & 86.99 & 0.82 & 511k \\
%     Winter barley & 86.29 & 85.10 & 84.56 & 0.83 & 164k \\
%     Winter rye    & 65.76 & 55.33 & 54.81 & 0.52 & 27k \\
%     Summer barley & 81.07 & 82.64 & 80.27 & 0.79 & 45k \\
%     Maize         & 95.84 & 90.78 & 92.63 & 0.88 & 572k \\
%     \hline
%     \textbf{Weighted Avg.} & 88.80 & 85.29 & 85.55 & - & - \\
%     \hline
%     \textbf{Overall Accuracy} & \multicolumn{5}{|c||}{91.84\%} \\
%     \textbf{Overall Kappa}    & \multicolumn{5}{|c||}{89.95\%} \\
%     \hline
%     \end{tabular}
%     \caption{Metriche delle prestazioni per la segmentazione delle colture sul dataset di 
%     validazione dell'anno 2016.}
%     \label{tab:crop_segmentation_2016_val}
% \end{table}

\begin{table}[H]
    \centering
    \setlength{\tabcolsep}{5pt} % Riduce la spaziatura tra le colonne
    \renewcommand{\arraystretch}{1.2} % Aumenta lo spazio tra le righe
    \begin{tabular}{|l||c|c|c|c||}
    \hline
    \multicolumn{5}{|c|}{\textbf{Validation}} \\
    \hline
    \textbf{Class} & \textbf{Precision} & \textbf{Recall} & \textbf{F1-Score} & \textbf{Kappa} \\
    \hline
    Sugar beet    & 86.37 & 93.51 & 88.85 & 0.88 \\
    Summer oat    & 69.55 & 75.66 & 70.12 & 0.69 \\
    Meadow        & 77.90 & 77.48 & 75.10 & 0.70 \\
    Rape          & 89.79 & 93.06 & 90.94 & 0.90 \\
    Hop           & 85.46 & 91.22 & 86.62 & 0.81 \\
    Winter spelt  & 71.31 & 63.33 & 63.45 & 0.60 \\
    Winter triticale & 66.92 & 55.66 & 55.99 & 0.53 \\
    Beans         & 76.44 & 84.39 & 78.35 & 0.74 \\
    Peas          & 82.81 & 79.52 & 78.31 & 0.77 \\
    Potato        & 84.58 & 85.88 & 83.77 & 0.81 \\
    Soybeans      & 81.53 & 83.95 & 80.61 & 0.78 \\
    Asparagus     & 79.17 & 78.92 & 75.87 & 0.72 \\
    Winter wheat  & 92.12 & 85.07 & 86.99 & 0.82 \\
    Winter barley & 86.29 & 85.10 & 84.56 & 0.83 \\
    Winter rye    & 65.76 & 55.33 & 54.81 & 0.52 \\
    Summer barley & 81.07 & 82.64 & 80.27 & 0.79 \\
    Maize         & 95.84 & 90.78 & 92.63 & 0.88 \\
    \hline
    \textbf{Weighted Avg.} & 88.80 & 85.29 & 85.55 & - \\
    \hline
    \textbf{Overall Accuracy} & \multicolumn{4}{|c||}{91.84\%} \\
    \textbf{Overall Kappa}    & \multicolumn{4}{|c||}{89.95\%} \\
    \hline
    \end{tabular}
    \caption{Metriche delle prestazioni per la segmentazione delle colture sul dataset di 
    validazione.}
    \label{tab:crop_segmentation_2016_val}
\end{table}
\newpage
Sul dataset di valutazione, il modello ha ottenuto i seguenti risultati: 

% \begin{table}[H]
%     \centering
%     \setlength{\tabcolsep}{5pt} % Riduce la spaziatura tra le colonne
%     \renewcommand{\arraystretch}{1.2} % Aumenta lo spazio tra le righe
%     \begin{tabular}{|l||c|c|c|c|c||}
%     \hline
%     \multicolumn{6}{|c|}{\textbf{Test}} \\
%     \hline
%     \textbf{Class} & \textbf{Precision} & \textbf{Recall} & \textbf{F1-Score} & \textbf{Kappa} & \textbf{\#Pixels} \\
%     \hline
%     Sugar beet      & 0.8113 & 0.9063 & 0.8464 & 0.8349 & 20k \\
%     Summer oat      & 0.7498 & 0.7672 & 0.7338 & 0.7205 & 21k \\
%     Meadow          & 0.7638 & 0.7816 & 0.7473 & 0.7134 & 129k \\
%     Rape            & 0.9045 & 0.9406 & 0.9167 & 0.8943 & 89k \\
%     Hop             & 0.8266 & 0.8509 & 0.8258 & 0.7822 & 60k \\
%     Winter spelt    & 0.7551 & 0.5876 & 0.5916 & 0.5361 & 27k \\
%     Winter triticale & 0.6536 & 0.5456 & 0.5546 & 0.5272 & 30k \\
%     Beans           & 0.7739 & 0.8244 & 0.7694 & 0.7541 & 13k \\
%     Peas            & 0.7095 & 0.7214 & 0.6916 & 0.6793 & 9k  \\
%     Potato          & 0.8080 & 0.8273 & 0.8072 & 0.7891 & 58k \\
%     Soybeans        & 0.8412 & 0.8782 & 0.8505 & 0.8042 & 13k \\
%     Asparagus       & 0.7382 & 0.6496 & 0.6014 & 0.5689 & 4k  \\
%     Winter wheat    & 0.9259 & 0.8567 & 0.8747 & 0.8313 & 528k \\
%     Winter barley   & 0.8976 & 0.8883 & 0.8799 & 0.8621 & 197k \\
%     Winter rye      & 0.5713 & 0.4768 & 0.4744 & 0.4544 & 14k \\
%     Summer barley   & 0.8290 & 0.8007 & 0.7871 & 0.7537 & 43k \\
%     Maize           & 0.9584 & 0.9120 & 0.9297 & 0.8752 & 640k \\
%     \hline
%     \textbf{Weighted Avg.} & 0.8973 & 0.8618 & 0.8652 & - & - \\
%     \hline
%     \textbf{Overall Accuracy} & \multicolumn{5}{|c||}{91.92\%} \\
%     \textbf{Overall Kappa}    & \multicolumn{5}{|c||}{89.82\%} \\
%     \hline
%     \end{tabular}
%     \caption{Metriche delle prestazioni per la segmentazione delle colture sul 
%     dataset di valutazione dell'anno 2016.}
%     \label{tab:crop_segmentation_2016_test}
% \end{table}

\begin{table}[H]
    \centering
    \setlength{\tabcolsep}{5pt} % Riduce la spaziatura tra le colonne
    \renewcommand{\arraystretch}{1.2} % Aumenta lo spazio tra le righe
    \begin{tabular}{|l||c|c|c|c||}
    \hline
    \multicolumn{5}{|c|}{\textbf{Test}} \\
    \hline
    \textbf{Class} & \textbf{Precision} & \textbf{Recall} & \textbf{F1-Score} & \textbf{Kappa} \\
    \hline
    Sugar beet      & 0.81 & 0.91 & 0.85 & 0.83 \\
    Summer oat      & 0.75 & 0.77 & 0.73 & 0.72 \\
    Meadow          & 0.76 & 0.78 & 0.75 & 0.71 \\
    Rape            & 0.90 & 0.94 & 0.92 & 0.89 \\
    Hop             & 0.83 & 0.85 & 0.83 & 0.78 \\
    Winter spelt    & 0.76 & 0.59 & 0.59 & 0.54 \\
    Winter triticale & 0.65 & 0.55 & 0.55 & 0.53 \\
    Beans           & 0.77 & 0.82 & 0.77 & 0.75 \\
    Peas            & 0.71 & 0.72 & 0.69 & 0.68 \\
    Potato          & 0.81 & 0.83 & 0.81 & 0.79 \\
    Soybeans        & 0.84 & 0.88 & 0.85 & 0.80 \\
    Asparagus       & 0.74 & 0.65 & 0.60 & 0.57 \\
    Winter wheat    & 0.93 & 0.86 & 0.87 & 0.83 \\
    Winter barley   & 0.90 & 0.89 & 0.88 & 0.86 \\
    Winter rye      & 0.57 & 0.48 & 0.47 & 0.45 \\
    Summer barley   & 0.83 & 0.80 & 0.79 & 0.75 \\
    Maize           & 0.96 & 0.91 & 0.93 & 0.88 \\
    \hline
    \textbf{Weighted Avg.} & 0.90 & 0.86 & 0.87 & - \\
    \hline
    \textbf{Overall Accuracy} & \multicolumn{4}{|c||}{91.92\%} \\
    \textbf{Overall Kappa}    & \multicolumn{4}{|c||}{89.82\%} \\
    \hline
    \end{tabular}
    \caption{Metriche delle prestazioni per la segmentazione delle colture sul 
    dataset di valutazione.}
    \label{tab:crop_segmentation_2016_test}
\end{table}


L'accuratezza nella classificazione delle colture ottenuta dal nostro modello 
può essere considerata molto soddisfacente, avendo raggiunto  
un valore di $91.84\%$ sulla validazione e un valore di $92.92\%$ sulla valutazione. 
Questi risultati superano l'accuratezza dell'$89.7\%$ ottenuta sul dataset di valutazione 
dal modello descritto nell'articolo "\textit{"Multi-Temporal Land Cover Classification with 
Sequential Recurrent Encoders"} \cite{ARTICOLO_ORIGINALE_MUNICH}, il primo a trattare 
l'applicazione di una rete neurale sul dataset Sentinel2-Munich480.
Il modello presentato in quell'articolo era basato su un'architettura di 
tipo Sequential Recurrent Encoders \cite{RETE_RICORRENTE} per la mappatura delle colture.
Il nostro modello ha ottenuto risultati comparabili a quelli descritti in tale articolo. 
In particolare, entrambi i modelli, presentano delle difficoltà nel riconoscere
le classi "\textit{Winter rye}" e "\textit{Winter triticale}", dovute principalmente 
alla difficoltà nel distinguere l'aspetto spettrale e fenologico delle due colture. 
% Questo dovuto principalmente alla difficoltà nel distinguere l'aspetto spettrale e 
% fenologico tra le due colture.
Inoltre, il nostro modello si è avvicinato anche ai risultati ottenuti dal modello descritto 
nell'articolo
"\textit{Enhancing Crop Segmentation in Satellite Image Time Series with Transformer Networks}" 
\cite{ARTICOLO_TRANSFORMER}. 
In questo articolo è stato presentato un modello basato su un'architettura 
\textit{Transformer} \cite{TRANSFORMER} per eseguire la mappatura delle 
colture sul dataset Sentinel2-Munich480, ottenendo una precisione del $96.14\%$ 
sul dataset di validazione e una precisione del $95.26\%$ sul dataset di valutazione.

\newpage
Confrontando il nostro modello con quelli trattati in altri articoli, questa è 
la situazione che si riscontra:


\begin{table}[H]
    \centering
    \setlength{\tabcolsep}{8pt} % Spaziatura tra le colonne
    \renewcommand{\arraystretch}{1.2} % Spaziatura tra le righe
    \begin{tabular}{|l||c|c|c|c|}
    \hline
    \multicolumn{5}{|c|}{\textbf{Munich Metrics}} \\
    \hline
    \textbf{Model} & \textbf{OA\_test} & \textbf{OK\_test} & \textbf{OA\_val} & \textbf{OK\_val} \\
    \hline
    Swin UNETR \cite{ARTICOLO_TRANSFORMER} & 95.26\% & 93.89\% & 96.14\% &  94.89\% \\
    UNet 3D \cite{UNET_3D_munich_application} & 94.73\% &  93.46\% & - & - \\
    FPN3D \cite{FPN3D}  & 93.11\% &  91.44\% & - & - \\
    *UNet 3D & 91.92\% & 89.82\% & 91.84\% & 89.95\% \\
    SRE \cite{ARTICOLO_ORIGINALE_MUNICH} & 89.6\% & 87.0\% & - & - \\
    DeepLabv3 3D \cite{ARTICOLO_TRANSFORMER} & 85.98\% &  82.53\% & - & - \\

    \hline
    \end{tabular}
    \caption{Confronto tra le metriche globali ottenute da differenti modelli. 
    OA rappresenta l'accuratezza complessiva e OK il coefficiente di kappa.   
    Il simbolo '*' indica il nostro modello.}
    \label{tab:munich_metrics}
\end{table}


%In ogni caso, possiamo 
Ci consideriamo soddisfatti dei risultati ottenuti, in quanto siamo riusciti a 
raggiungere gli obiettivi prefissati. Tuttavia, esistono ancora margini di miglioramento 
per ottimizzarne ulteriormente l'accuratezza del nostro modello. Ad esempio, 
si potrebbe prestare maggiore attenzione a come variare il valore del \textit{learning rate}, per 
evitare che il modello resti bloccato in un minimo locale, così come alla scelta 
degli algoritmi di ottimizzazione da utilizzare.
Inoltre, alcuni aspetti legati all’architettura della rete potrebbero essere perfezionati. 
Nello specifico, in questo lavoro è stata impiegata una nostra variante dell’architettura 
U-Net 3D, diversa rispetto ad altre implementazioni presenti in diversi repository 
\cite{Implementazione1_3D_UNET,Implementazione2_3D_UNET,UNET_3D_munich_application}.







% L'accuratezza ottenuta dalla classificazione delle coltura la possiamo definire più 
% che buona, in quanto siamo riuscita ad ottenere un valore di accuratezza 
% sui dati del 2016 pari a $91.7\%$. Superando l'accuratezza del $89.7\%$ ottenuta dal 
% modello descritto nell'articolo "\textit{Multi-Temporal Land Cover Classification with
% Sequential Recurrent Encoders}", il primo articolo che ha trattato il dataset di 
% sentinel2-Munich480, che si basava su una rete ricorrente per 
% eseguire la mappature delle culture \cite{ARTICOLO_ORIGINALE_MUNICH}. 
% In oltre, siamo riusciti anche ad avvicinarsi all'accuratezza del $96.14\%$ raggiunta dal 
% modello esposto nell'articolo "\textit{Enhancing Crop Segmentation in 
% satellite image time series with Transformer Networks}", che sfruttava una rete 
% \textit{Transformer} per eseguire la mappature delle culture \cite{ARTICOLO_TRANSFORMER}.

% Questo valore di accuratezza può ancora essere leggermente migliorato prestando una 
% maggiore attenzione ad esempio al \textit{learning rate}, agli algoritmi di ottimizzazione 
% o a come vengono rappresentati i dati. Oppure prestando anche attenzione ad alcune aspetti 
% dell'architettura della rete, in quanto abbiamo utilizzano una nostra variazione 
% dell'architettura dell'Unet 3D e non le stesse identiche implementazioni 
% \cite{Implementazione1_3D_UNET,Implementazione2_3D_UNET}.


% \begin{table}[ht]
%     \centering
%     \setlength{\tabcolsep}{5pt} % Riduce la spaziatura tra le colonne
%     \renewcommand{\arraystretch}{1.2} % Aumenta lo spazio tra le righe
%     \begin{tabular}{|l|c|c|c|c|c|c||c|c|c|c|c|c|}
%     \hline
%     \textbf{Class} & \multicolumn{6}{|c||}{\textbf{2016}} & \multicolumn{6}{|c|}{\textbf{2017}} \\
%     \hline
%     & Precision & Recall & $f$-Meas. & Kappa & \#Pixels & & Precision & Recall & $f$-Meas. & Kappa & \#Pixels \\
%     \hline
%     Sugar beet    & 94.6 & 77.6 & 85.3 & .772 & 59k  & & 89.2 & 78.5 & 83.5 & .779 & 94k  \\
%     Oat           & 86.1 & 67.8 & 75.8 & .675 & 36k  & & 63.8 & 62.8 & 63.3 & .623 & 38k  \\
%     Meadow        & 90.8 & 85.7 & 88.2 & .845 & 233k & & 88.1 & 85.0 & 86.5 & .837 & 242k \\
%     Rapeseed      & 95.4 & 90.0 & 92.6 & .896 & 125k & & 96.2 & 95.9 & 96.1 & .957 & 114k \\
%     Hop           & 96.4 & 87.5 & 91.7 & .873 & 51k  & & 92.5 & 74.7 & 82.7 & .743 & 53k  \\
%     Spelt         & 55.1 & 81.1 & 65.6 & .807 & 38k  & & 75.3 & 46.7 & 57.6 & .463 & 31k  \\
%     Triticale     & 69.4 & 55.7 & 61.8 & .549 & 65k  & & 62.4 & 57.2 & 59.7 & .563 & 64k  \\
%     Beans         & 92.4 & 87.1 & 89.6 & .869 & 27k  & & 92.8 & 63.2 & 75.2 & .620 & 28k  \\
%     Peas          & 93.2 & 70.7 & 80.4 & .706 & 9k   & & 60.9 & 41.5 & 49.3 & .414 & 6k   \\
%     Potato        & 90.9 & 88.2 & 89.5 & .795 & 126k & & 95.2 & 73.8 & 83.1 & .728 & 140k \\
%     Soybeans      & 97.7 & 79.6 & 87.7 & .795 & 21k  & & 75.9 & 79.9 & 77.8 & .798 & 26k  \\
%     Asparagus     & 89.2 & 78.8 & 83.7 & .787 & 20k  & & 81.6 & 77.5 & 79.5 & .773 & 19k  \\
%     Wheat         & 87.7 & 93.1 & 90.3 & .902 & 806k & & 90.1 & 95.0 & 92.5 & .915 & 783k \\
%     Winter barley & 95.2 & 87.3 & 91.0 & .861 & 258k & & 92.5 & 92.2 & 92.4 & .915 & 255k \\
%     Rye           & 85.6 & 47.0 & 60.7 & .466 & 43k  & & 76.7 & 61.9 & 68.5 & .616 & 30k  \\
%     Summer barley & 87.5 & 83.4 & 85.4 & .830 & 73k  & & 77.9 & 88.5 & 82.9 & .880 & 91k  \\
%     Maize         & 91.6 & 96.3 & 93.9 & .944 & 919k & & 92.3 & 96.8 & 94.5 & .953 & 876k \\
%     \hline
%     \textbf{Weight avg.} & 89.9 & 89.7 & 89.5 & & & & 89.5 & 89.5 & 89.5 & & \\
%     \hline
%     \textbf{Overall Accuracy} & \multicolumn{6}{|c||}{89.7} & \multicolumn{6}{|c|}{89.5} \\
%     \textbf{Overall Kappa}    & \multicolumn{6}{|c||}{.870} & \multicolumn{6}{|c|}{.870} \\
%     \hline
%     \end{tabular}
%     \caption{Performance metrics for crop segmentation over 2016 and 2017.}
%     \label{tab:crop_segmentation}
%     \end{table}


    
    