% Abstract
\chapter*{Introduzione}
\addcontentsline{toc}{chapter}{Introduzione} 

Il telerilevamento è una scienza che permette di identificare, misurare ed
analizzare le caratteristiche qualitative e quantitative di un determinato oggetto,
area o fenomeno senza entrare in contatto diretto con esso. In generale l’oggetto di
studio nel telerilevamento è il pianeta Terra in tutte le sue componenti: territorio,
acqua e atmosfera. Avendo la possibilità di operare dall'alto, a diverse distanze
e in tempi differenti, questa disciplina ha introdotto una nuova filosofia di
controllo e d’indagine nello studio del territorio e dei relativi problemi,
permettendo di osservare fenomeni non direttamente accessibili e quindi di
superare le difficoltà connesse alle campagne di misura a terra, quali grandi sforzi
organizzativi, tempo e risorse non sempre disponibili 
\cite{ALL4_REMOTE_SENSING, ALL5_REMOTE_SENSING,ALL6_REMOTE_SENSING}.

Negli ultimi anni, in quest’ambito si è fatta molta strada in termini tecnologici
e le risoluzioni, sia geometriche che spettrali, dei sensori impiegati, sono
nettamente migliorate, permettendo di estendere le applicazioni del
telerilevamento all’agricoltura di precisione. L'agricoltura di precisione è una strategia di 
gestione dell’attività agricola con la quale i dati vengono raccolti, 
elaborati, analizzati e combinati con altre informazioni per orientare le decisioni in 
funzione della variabilità spaziale e temporale al fine di migliorare l'efficienza 
nell'uso delle risorse, la produttività, la qualità, la redditività e la sostenibilità 
della produzione agricola \cite{Agricoltura_precisione}.

La mappatura delle colture (\textit{Crop mapping}) è fondamentale per supportare il 
processo decisionale e fornire inventari accurati e tempestivi per stimare la 
produzione e monitorare la crescita dinamica delle colture a varie scale. 
Tuttavia, la mappatura delle colture in situ (sul posto) si rivela spesso costosa e ad alta 
intensità di lavoro. Il telerilevamento satellitare offre un'alternativa più economica, 
in grado di fornire serie di dati temporali in grado di catturare ripetutamente le 
dinamiche della crescita delle colture su larga scala e a intervalli regolarmente 
rivisitati. Sebbene la maggior parte dei prodotti di tipo colturale esistenti sia 
generata utilizzando dati di telerilevamento e approcci di apprendimento automatico, 
l'accuratezza delle previsioni può essere bassa, dato che persistono errori di classificazione 
dovuti alle somiglianze fenologiche tra le diverse colture e alla complessità dei sistemi agricoli 
negli scenari reali. Le reti neurali profonde dimostrano un grande potenziale nel catturare i 
modelli stagionali e le relazioni sequenziali nei dati delle serie temporali nel contesto 
del loro modo di apprendere le caratteristiche in modo completo e autonomo. 

% Questa tesi ha presentato 
% un'esplorazione completa di metodologie avanzate di deep learning per la mappatura 
% delle colture agricole su larga scala, utilizzando dati di telerilevamento multi-temporali e multi-sorgente.


Questa tesi esplora le applicazioni delle reti neurali, nello specifico di reti 
convoluzionali (CNN), per affrontare il problema della segmentazione semantica 
su immagini satellitari Sentinel-2, con l'obiettivo principale di mappare su 
larga scala le colture agricole.
%Al fine della mappatura delle colture agricole su larga scala.
La tesi espone lo sviluppo di un sistema capace di apprendere 
automaticamente le caratteristiche di ogni campo agricolo, utilizzando per l’addestramento 
il dataset Sentinel2-Munich480 \cite{Munich480}.
Questo sistema, applicabile alle immagini satellitari, 
consente di identificare i campi agricoli e di riconoscere le diverse colture.
% In modo che possa essere utilizzato su 
% immagini satellitari, per riconoscere, tramite segmentazione segmentazione semantica, i campi 
% agricoli e distinguere il contenuto. Utilizzando per lo sviluppo del sistema i dati del dataset 
% Sentinel2-Munich480 \cite{Munich480}.
% L’obiettivo della tesi consisterebbe nello sviluppo di un sistema capace di apprendere 
% automaticamente le caratteristiche di ogni campo agricolo. In modo che possa essere utilizzato 
% per riconoscere e distinguere, tramite segmentazione segmentazione semantica, i campi agricoli 
% presenti nelle immagini satellitari.
La tesi espone anche tutta la parte teorica relativa al funzionamento delle reti neurali, sia classiche 
che convolutive, esponendo anche come vengono rappresentati e acquisiti i dati nel telerilevamento. 
% Nel corso degli esperimenti, verranno addestrate e valutate diverse architetture 
% di reti neurali, esplorando le potenzialità delle diverse architetture nella 
% cattura di informazioni spaziali e temporali.

% Illustrando anche tutto quello che c'è dietro al funzionamento 
% delle reti neurali.

La tesi è strutturata su otto capitoli:
Il primo capitolo tratta la definizione di rete neurale, di \textit{Deep learning} e di 
\text{Machine Learning}.
Viene inoltre illustrato il concetto di dataset ed i metodi che si utilizzano per 
valutare le prestazioni di un modello di \textit{Deep learning}.

Nel secondo capitolo vengono presentati i framework che permettono lo sviluppo di modelli 
di \textit{Deep learning}, il linguaggio di programmazione utilizzato per la realizzazione dei 
modelli ed il concetto di Tensore.

Nel terzo capitolo vengono richiamati i concetti fondamentali del
telerilevamento, illustrando come avviene l'acquisizione delle informazioni nel telerilevamento.

Nel quarto capitolo si approfondisce il funzionamento di una rete neurale classica, 
partendo dalle basi fino a trattare concetti più complessi.

Nel quinto capitolo vengono trattate le reti neurali convolutive, illustrando i principi 
su cui si basano queste tipologie di reti e gli elementi di cui sono composte. 

% Nel sesto capitolo vengono richiamati i concetti di base dietro alla segmentazione delle immagini.
% Sempre nello stesso capitolo, viene anche illustrata l'architettura UNet.
Nel sesto capitolo vengono trattati i concetti di base relativi alla segmentazione 
delle immagini. Illustrando anche l'architettura UNet, spesso utilizzata per applicazioni 
di segmentazione semantica delle immagini.

Il settimo capitolo tratta la sperimentazione, illustrando il processo per la realizzazione di 
un modello in grado di eseguire la segmentazione semantica su delle immagini satellitari.

% Mentre nell'ottavo capitolo vengono discussi i risultati ottenuti, mettendoli a confronto con i 
% risultati ottenuti da altri modelli applicati allo stesso problema. 
Nell'ottavo capitolo vengono discussi i risultati ottenuti, confrontandoli
con quelli ottenuti da altri modelli descritti in articoli che utilizzano lo stesso dataset.